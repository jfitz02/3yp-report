\chapter{Introduction}
\label{ch:introduction}

Bias in social media is prevelant no matter where you look. Whether it's echo chambers causing a user to only interact with
people who agree with them, or the same style of posts being shown to a user, there is a lot of bias in social media.\\

This project aims to explore a specific kind of bias in social media: Topical bias. Topical bias is when certain topics
are overrepresented in a users social media feed. But what is a topic? Broadly speaking, a topic is an identifier
of a set of related posts. For example, the topic "cats" would include posts about cats, cat videos, cat memes, etc.\\

The first focus of this project is to explore current methods of topic identification in social media and use
these methods to identify topical bias. The second focus of this project is to analyse the effect of this bias over time.
We can represent the bias as a n-dimensional vector and analyse the velocity/acceleration of the vector over time. The aim
will be to identify how different social media "strategies" affect the bias vector over time - A "strategy" is just a way in
which a user uses social media: Just scrolls (passive), likes all posts (active), likes specific topics (specified), etc.

\section{Related work}

\section{Objectives}

This project has 2 goals. The first is to implement a method for identifying topics in social media posts. The second is to
Analyse the effect of topical bias over time with different "strategies".

\begin{itemize}
    \item Generate a list of topics I want to analyse
    \item Implement a method for identifying the above topics in social media posts
    \item Measure the accuracy of this methods
    \item 
\end{itemize}