\usepackage{rotating}

\section{Project management}

\subsection{Initiation}
The goal for the initiation of this project was to show the need for the project as well as setup
a strong plan for the project. This is best achieved by creating a Project Initiation Document (PID).\\
A PID should contain the following information:
\begin{itemize}
    \item Project overview
    \item Project objectives
    \item Project scope
    \item Project deliverables
    \item Project constraints
    \item Project assumptions
    \item Project risks
    \item Project budget
    \item Project schedule
\end{itemize}\cite{}
The specification document was created to fulfil the requirements of the PID, and can be found in the appendix.

\subsection{Planning}
The planning for this project was also completed in the specification document. The method used for creating the plan was
not optimal. This is due to the fact it was made completely on guess work. Although it is common practice for developers
to take into account their previous experiences when creating a plan, it is not the best approach for this project due to
the lack of experience available. This was ultimately the reason for the plan being changed during the project; the plan
was too fast paced and did not allow for enough time to complete the tasks.\\

The plan could have been further improved by creating a Work Breakdown Structure (WBS). This would have given a broken
down view of the project and have allowed for better estimation of the time required for work package.\\
For the updated plan, a WBS was created.


% rotate figure 90 degrees anti-clockwise
\begin{figure}[htbp]
    \centering
    \includegraphics[angle=90, width=0.5\textwidth]{../images/wbs.png}
    \caption{Work Breakdown Structure}
    \label{fig:wbs}
\end{figure}

\newpage
    

% Include WBS here

\subsection{Execution}
An agile approach has been chosen for the execution of this project. This is due to the fact that the project involves
a lot of experimentation. The initial stages of implementation required setting up different neural networks and
comparing their performance. This is best achieved by using an agile approach as agile allows for changes during development \cite.\\

The chosen agile approach is an individual version of Scrum. A set of goals is established every 2 weeks and then the progress
is evaluated at the end of the 2 weeks. This is most evidently shown by section \ref{sec:progress} of this report.\\

Implementation was only started recently. The plan is to stick to an agile approach for the model development. The data analysis
part of the project will be completed using a waterfall approach. Firstly, a plan will be developed and then the analysis will be
performed.
\subsection{Timetable}